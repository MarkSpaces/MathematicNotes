% \documentclass[a4paper, 11pt, oneside, onecolumn, titlepage]{ctexrep}

% \usepackage{makeidx,amsmath}
% \usepackage{unicode-math}
% \usepackage{setspace}
% \usepackage{amsthm}
% \usepackage{geometry}
% \usepackage{fancyhdr}
% \usepackage{tikz}

% \geometry{left=25mm,right=25mm,top=30mm,bottom=30mm}
% \setmathfont{XITS Math}

% \linespread{1.5}
% \title{Numerical Integration \& Numerical Differentiation}
% \author{Brassicaaa}
% \date{\today}

% \setlength{\headheight}{27pt}

% \begin{document}
% \pagestyle{fancy}
% \lhead{Notes for Numerical Analysis}
% \setcounter{page}{1}
% \pagenumbering{arabic}


\chapter{数值积分与数值微分}
\section{数值积分的概念}

微积分中的积分定义:
\begin{equation}
\int_a^b f(x)\,\mathrm{d}x = \lim\limits_{\substack{n\to \infty\\ \max \Delta x_k \to 0}}\sum\limits_{k=1}\limits^n \Delta x_k f(x_k)
\end{equation}

Newton-Leibniz公式:
\begin{equation}
I = \int_a^b f(x)\,\mathrm{d}x = F(b) - F(a)
\end{equation}

数值积分的基本思想是将其转化为\textbf{函数值的线性组合}:
\begin{align}
\int_a^b f(x)\,\mathrm{d}x &= \lim\limits_{\substack{n\to \infty\\ \max \Delta x_k \to 0}}\sum\limits_{k=1}\limits^n \Delta x_k f(x_k) \approx \sum\limits_{k=1}\limits^n \Delta x_k f(x_k)\notag \\
&\overset{\text{推广}}{\approx} \sum\limits_{k=0}\limits^n A_k f(x_k) \quad \text{(机械求积法)}
\end{align}

\textbf{代数精度用来衡量数值求积公式的准确程度},若某个求积公式对于次数$\leqslant m$的代数多项式都能精确成立,但对$m+1$次代数多项式不一定精确成立,则称该求积公式具有$m$次\textbf{代数精度},即:
\begin{equation}\begin{cases}
\ f(x)=x^\mu,\quad &\sum\limits_{k=0}\limits^n A_k x_k^\mu = \frac{1}{\mu+1}(b^{\mu+1}-a^{\mu+1})\quad (\mu \leqslant m) \\[4mm]
\ f(x)=x^{m+1},\quad &\sum\limits_{k=0}\limits^n A_k x_k^{m+1} \neq \frac{1}{m+2}(b^{m+2}-a^{m+2})
\end{cases}
\end{equation}
\newpage


\section{插值型数值积分公式}

\subsection{Lagrange Integral Formula}
根据Lagrange插值:
\begin{equation}\label{l1}
\int_a^b f(x)\,\mathrm{d}x \approx \int_a^b \sum\limits_{k=0}\limits^n l_k(x)f(x_k)\,\mathrm{d}x = \sum\limits_{k=0}\limits^n [\int_a^b l_k(x)\,\mathrm{d}x]f(x_k)= \sum\limits_{k=0}\limits^n A_k f(x_k)
\end{equation}

该式称为\textbf{插值型求积公式},其中:
$$
A_k = \int_a^b l_k(x)\,\mathrm{d}x
$$

插值型求积公式的余项:
\begin{equation}
R(f) = \int_a^b [f(x) - L_n(x)]\,\mathrm{d}x = \int_a^b \frac{f^{(n+1)}(\xi)}{(n+1)!}\omega_{n+1}(x)\,\mathrm{d}x\quad \xi = \xi(x)\in [a,b]
\end{equation}

\newtheorem{thm3}{Theorem}[chapter]
\begin{thm3}
形如公式\ref{l1} 的求积公式至少具有$n$次代数精度的充要条件为该求积公式是插值型的。
\end{thm3}


\subsection{Equidistent Integral Formula}
\begin{enumerate}
\item \textbf{Newton-Cotes公式}

设求积区间$[a,b]$分为$n$等份,步长$h=\frac{b-a}{n}$,选取等距节点$x_k=a+kh(k=0,1,\dots,n)$,构造插值型求积公式:
\begin{align}
I_n &= \sum\limits_{k=0}\limits^n [\int_a^b l_k(x)\,\mathrm{d}x] f(x_k) 
= \sum\limits_{k=0}\limits^n [\int_a^b \prod\limits_{\substack{j=0\\ j\neq k}}\limits^n\frac{x-x_j}{x_k-x_j} \,\mathrm{d}x] f(x_k) \notag \\[2mm]
&= \sum\limits_{k=0}\limits^n [h \int_0^n \prod\limits_{\substack{j=0\\ j\neq k}}\limits^n\frac{t-j}{k-j} \,\mathrm{d}t] f(x_k) 
= \sum\limits_{k=0}\limits^n [\frac{h}{\prod\limits_{\substack{j=0\\ j\neq k}}\limits^n(k-j)}\int_0^n \prod\limits_{\substack{j=0\\ j\neq k}}\limits^n  (t-j) \,\mathrm{d}t] f(x_k)\notag \\[2mm]
&= \sum\limits_{k=0}\limits^n [\frac{h(-1)^{n-k}}{k!(n-k)!}\int_0^n \prod\limits_{\substack{j=0\\ j\neq k}}\limits^n  (t-j) \,\mathrm{d}t] f(x_k)\notag \\[2mm]
&= (b-a)\sum\limits_{k=0}\limits^n c_k^{(n)} f(x_k) 
\end{align}

称作\textbf{Newton-Cotes公式},式中$c_k^{(n)}$称作Cotes系数:
\begin{equation}
c_k^{(n)} = \frac{h}{b-a}\int_0^n\prod\limits_{\substack{j=0\\ j\neq k}}\limits^{n}\frac{t-j}{k-j}\,\mathrm{d}t
= \frac{(-1)^{n-k}}{nk!(n-k)!}\int_0^n\prod\limits_{\substack{j=0\\ j\neq k}}\limits^{n}(t-j)\,\mathrm{d}t 
\end{equation}

并有:
\begin{equation}
A_k = (b-a)c_k^{(n)}
\end{equation}

Cotes系数的性质:
\begin{enumerate}
\item 当$\mu = 0$时,有$\sum\limits_{k=0}\limits^n A_k = b-a$,结合$A_k = (b-a) c_k^{(n)}$知:
\begin{equation}
\sum\limits_{k=0}\limits^n c_k^{(n)} = 1
\end{equation}

\item 利用$c_k^{(n)}$的表达式,并作代换$u=n-t$可证明:
\begin{equation}
c_k^{(n)} = c_{n-k}^{(n)}
\end{equation}

\item Cotes系数有正有负,一般不用高阶Newton-Cotes公式,与不用高阶Lagrange插值同理。
\end{enumerate}

将$x=a+th$代入插值型积分公式余项得:
\begin{equation}
R(f) = \int_a^b \frac{f^{(n+1)}(\xi)}{(n+1)!}\omega_{n+1}(x)\,\mathrm{d}x\quad \xi = \xi (x)\in[a,b]
\end{equation}
显然,当$f(x)=x^n$时,$f^{(n+1)}(\xi)=0,\ R(f)=0$,即该积分公式具有$n$次代数精度,本身也是插值型积分公式。特别地,当$n$为偶数时,$f^{(n+1)}(\xi)=(n+1)!$,有:
$$
R(f) = h^{n+2}\int_0^n\prod\limits_{j=0}\limits^{n}(t-j)\,\mathrm{d}t
$$

令$t = u + \frac{n}{2}$,
$$
R(f) = h^{n+2}\int_{- \frac{n}{2}}^{\frac{n}{2}}\prod\limits_{j=0}\limits^{n}(u+\frac{n}{2}-j)\,\mathrm{d}u 
$$

对被积函数:
$$
F(u) = \prod\limits_{j=0}\limits^{n}(u+\frac{n}{2}-j) = u \prod\limits_{i=1}\limits^{\frac{n}{2}} [u^2 - i^2] 
$$
则$F(-u)=-F(u)$,即$F(u)$为奇函数,故$R(f)=0$,表明当$n$为偶数时,Newton-Cotes公式具有$n+1$次代数精度。

\begin{thm3}[积分中值定理]
对$f(x)\in C[a,b]$,$g(x)$在区间$[a,b]$上可积且保号,则:
\begin{equation}
\int_a^b f(x)g(x)\,\mathrm{d}x = f(\eta)\int_a^b g(x)\,\mathrm{d}x\quad \eta\in[a,b]
\end{equation}
\end{thm3}

特别地,考虑如下特殊情形:
\begin{enumerate}
\item 当$n=1$时,
$$
c_0^{(1)} = c_1^{(1)} = \frac{1}{2}
$$

对应\textbf{梯形公式}:
\begin{equation}
I_1 \triangleq T = \frac{b-a}{2}[f(a)+f(b)]
\end{equation}

当$f^{\prime \prime}(x)$在区间$[a,b]$上连续时,梯形公式的余项:
\begin{align}
R_T &= \int_a^b  \frac{f^{\prime \prime}(\xi)}{2!}(x-a)(x-b)\,\mathrm{d}x \notag \\[4mm]
&= \frac{f^{\prime \prime}(\eta)}{2!}\int_a^b(x-a)(x-b)\,\mathrm{d}x\quad \text{\small{(积分中值定理)}} \notag \\[4mm]
&= -\frac{f^{\prime \prime}(\eta)}{12}(b-a)^3\quad \eta \in [a,b]
\end{align}

\item 当$n=2$时:
$$
c_0^{(2)} = \frac{1}{6}\quad c_1^{(2)} = \frac{4}{6}\quad c_2^{(2)} = \frac{1}{6}
$$

对应\textbf{Simpson公式}:
\begin{equation}
I_2 \triangleq S = \frac{b-a}{6}[f(a)+4f(\frac{a+b}{2})+f(b)]
\end{equation}

Simpson公式的余项:因为其具有三次代数精度,直接利用余项公式不能完全反映误差,构造次数$\leqslant 3$的多项式$H(x)$,满足插值条件:
$$
H(a)=f(a),\quad H(b)=f(b),\quad H(\frac{a+b}{2}) = f(\frac{a+b}{2}),\quad H^{\prime}(\frac{a+b}{2})=f^{\prime}(\frac{a+b}{2})
$$
由于其对三次多项式$H(x)$同样精确成立,有$\int_a^b H(x)\,\mathrm{d}x = S$,假设$f(x)\in C^4 [a,b]$,则余项为:
\begin{align}
R_s &= I-H = \int_a^b [f(x)-H(x)]\,\mathrm{d}x \notag \\[4mm]
&= \int_a^b \frac{f^{(4)}(\xi)}{4!}(x-a)(x-\frac{a+b}{2})^2(x-b) \,\mathrm{d}x\quad  \text{\small{(非标Hermite余项)}}\notag \\[4mm]
&= \frac{f^{(4)}(\eta)}{4!}\int_a^b(x-a)(x-\frac{a+b}{2})^2(x-b)\,\mathrm{d}x\quad \text{\small{(积分中值定理)}}\notag \\[4mm]
&= -\frac{b-a}{180}(\frac{b-a}{2})^4f^{(4)}(\eta)\quad \eta \in [a,b]
\end{align}
注意,仅当取$H^{\prime}(\frac{a+b}{2})=f^{\prime}(\frac{a+b}{2})$时满足保号(非正)且可积,才满足积分中值定理条件,取$a$或$b$点导数值时均不满足保号性,并且在高阶时同样需要考虑保号性。

\item 当$n=4$时:
$$
c_0^{(4)} = c_4^{(4)} = \frac{7}{90},\quad c_1^{(4)} = c_3^{(4)} = \frac{16}{45},\quad c_2^{(4)} = \frac{2}{15}
$$

对应\textbf{Cotes公式}:
\begin{equation}
I_4 \triangleq C = \frac{b-a}{90}[7f(x_0)+32f(x_1)+12f(x_2)+32f(x_3)+7f(x_4)]
\end{equation}

Cotes公式的余项:
\begin{equation}
R_C = I-C = -\frac{2(b-a)}{945}(\frac{b-a}{4})^6 f^{(6)}(\eta)\quad \eta \in [a,b]
\end{equation}
\end{enumerate}

\textbf{数值稳定性指的是舍入误差对计算结果产生的影响。}假设$f(x_k)$有舍入误差$\varepsilon_k$,产生误差记为$e_n$,当Cotes系数全为正时,对Newton-Cotes公式有:
\begin{align}
e_n &= \left| (b-a)\sum\limits_{k=0}\limits^{n} c_k^{(n)} [f(x_k)+\varepsilon_k] - (b-a)\sum\limits_{k=0}\limits^{n} c_k^{(n)} f(x_k)\right| \notag \\[4mm]
&= (b-a)\left| \sum\limits_{k=0}\limits^{n}c_k^{(n)}\varepsilon_k \right| 
\leqslant (b-a) \sum\limits_{k=0}\limits^{n}\left| c_k^{(n)}\right| \cdot \left| \varepsilon_k \right| \notag \\[4mm]
&\leqslant \varepsilon (b-a)\sum\limits_{k=0}\limits^{n}c_k^{(n)}= (b-a)\varepsilon\quad
(\varepsilon = \max\left| \varepsilon_k \right|,\ c_k^{(n)}> 0,\ k=0,1,\dots,n)\notag
\end{align}



\item \textbf{复化求积法}

类似于分段低次插值的思想,复化求积法的基本思想是将积分区间分成若干个小区间,在每个小区间上采用低次插值多项式来代替被积函数积分,然后加起来得到整个区间上的求积公式。

\begin{thm3}[最值定理]
如果$f(x)\in C[a,b]$,则对$\forall x\in[a,b]$,必然存在$m$、$M\ (m\leqslant M)$满足:
$$
m \leqslant f(x) \leqslant M
$$
\end{thm3}

\begin{thm3}[介值定理]
如果$f(x)\in C[a,b],\ f(x)_{min}=m,\ f(x)_{max}=M\ (m<M)$,则必然$\exists x_0\in[a,b]$,满足:
$$
m < f(x_0) < M
$$
\end{thm3}

\begin{enumerate}
\item 复化梯形公式为:
\begin{align}
T_n &= \sum\limits_{k=0}\limits^{n-1} \frac{h}{2}[f(x_k)+f(x_{k+1})] \notag \\[2mm]
&= \frac{h}{2}[f(a) + 2\sum\limits_{k=1}\limits^{n-1}f(x_{k})+ f(b)] \\[6mm]
R_{T_n} &= I-T_n = \sum\limits_{k=0}\limits^{n-1} [-\frac{h^3}{12}f^{\prime \prime}(\eta_k)] \notag \\[2mm]
&= -\frac{b-a}{12}h^2f^{\prime \prime}(\eta)\quad \eta \in [a,b]
\quad \text{\small{(介值定理)}}
\end{align}

\item 记区间$[x_k,x_{k+1}]$的中点为$x_{k+\frac{1}{2}}$,则复化Simpson公式为:
\begin{align}
S_n &= \sum\limits_{k=0}\limits^{n-1} \frac{h}{6} [f(x_k) + 4f(x_{k+\frac{1}{2}}) + f(x_{k+1})] \notag \\[2mm]
&= \frac{h}{6}[f(a) + 4\sum\limits_{k=0}\limits^{n-1}f(x_{k+\frac{1}{2}}) + 2\sum\limits_{k=1}\limits^{n-1}f(x_k) + f(b)] \\[6mm]
R_{S_n} &= I-S_n = \sum\limits_{k=0}\limits^{n-1}(-\frac{h}{180})(\frac{h}{2})^4f^{(4)}(\eta_k)\notag \\[2mm]
&= -\frac{b-a}{180} (\frac{h}{2})^4 f^{(4)} (\eta)\quad \eta \in [a,b]
\end{align}
\newpage

\item 将区间$[x_k,x_{k+1}]$进行4等分,内分点依次记作$x_{k+\frac{1}{4}},\ x_{k+\frac{1}{2}},\  x_{k+\frac{3}{4}}$,则复化Cotes公式为:
\begin{align}
C_n = &\frac{h}{90}[7f(a) + 32\sum\limits_{k=0}\limits^{n-1}f(x_{k+\frac{1}{4}}) + 12\sum\limits_{k=0}\limits^{n-1}f(x_{k+\frac{1}{2}}) \notag \\
&+ 32\sum\limits_{k=0}\limits^{n-1}f(x_{k+\frac{3}{4}}) + 14\sum\limits_{k=1}\limits^{n-1}f(x_k) + 7f(b)] \\[4mm]
R_{C_n} = &I-C_n = -\frac{2(b-a)}{945}(\frac{h}{4})^6f^{(6)}(\eta) \quad \eta \in [a,b]
\end{align}
\end{enumerate}

\textbf{定义:}若一种复化求积公式$I_n$在$h\to 0$时有渐进关系式:
$$
\frac{I-I_n}{h^p}\to C \text{(C为常数)}
$$
则称该求积公式$I_n$是$p$阶收敛的。

复化梯形公式:
\begin{align}
&\frac{I-T_n}{h^2} = -\frac{1}{12}\sum\limits_{k=0}\limits^{n-1}hf^{\prime \prime}(\eta_k)\overset{h\to 0}{\longrightarrow} -\frac{1}{12}\int_a^b f^{\prime \prime}(x)\,\mathrm{d}x \notag \\[4mm]
\Rightarrow \  &\lim\limits_{h\to 0}\frac{I-T_n}{h^2} = -\frac{1}{12}[f^{\prime}(b) - f^{\prime}(a)] \notag \\[4mm]
\Rightarrow \  &I-T_n\approx  -\frac{h^2}{12}[f^{\prime}(b) - f^{\prime}(a)]\notag \\[4mm]
\text{\small{同理可得:}}\ &I-S_n\approx  -\frac{1}{180}(\frac{h}{2})^4[f^{\prime \prime \prime}(b) - f^{\prime \prime \prime}(a)]\notag \\[4mm]
&I-C_n\approx  -\frac{2}{945}(\frac{h}{4})^6[f^{(5)}(b) - f^{(5)}(a)] \notag
\end{align}

表明复化梯形公式、复化Simpson公式和复化Cotes公式分别具有2阶、4阶和6阶收敛性,若将步长减半,其误差分别为原有误差的$\displaystyle{\frac{1}{4}}$,$\displaystyle{\frac{1}{16}}$和$\displaystyle{\frac{1}{64}}$.
\newpage


\textbf{自适应步长的选取}:为避免事先给出积分步长这一困难,实际计算时将区间逐次二分,反复利用复化求积公式计算,直至二分前后的两次积分近似值之差符合精度要求为止。

对复化梯形公式,将$[a,b]$区间作$n$等分,计算$T_n$时需要$n+1$个点的函数值。若将求积区间再次二分,新增$n$个节点,记区间$[x_k,x_{k+1}]$上的新增节点为$x_{k+\frac{1}{2}}=\frac{1}{2}(x_k + x_{k+1})\ (k=0,1,\dots,n-1)$,则该区间上的积分近似值为:
\begin{align}
&T_k = \frac{h}{2}[f(x_k)+f(x_{k+1})] \notag \\[2mm]
\Rightarrow \ &T_k^{\prime} = \frac{h}{4}[f(x_k) + 2f(x_{k+\frac{1}{2}}) + f(x_{k+1})]\notag
\end{align}

将每个子区间上的积分近似值求和,得到递推公式:
\begin{align}
T_{2n} &= \frac{h}{4}\sum\limits_{k=0}\limits^{n-1}[f(x_k)+f(x_{k+1})] + \frac{h}{2}\sum\limits_{k=0}\limits^{n-1} f(x_{k+\frac{1}{2}}) \notag \\[2mm]
&= \frac{1}{2}T_n + \frac{h}{2}\sum\limits_{k=0}\limits^{n-1} f(x_{k+\frac{1}{2}})
\end{align}

当$f^{\prime \prime}(x)$在$[a,b]$上变化不大时$(\eta_n \approx \eta_{2n})$,由复化梯形公式的误差估计式得:
\begin{align}
&\frac{R_{T_n}}{R_{T_{2n}}} = \frac{I-T_n}{I-T_{2n}} = \frac{{\displaystyle -\frac{b-a}{12}h^2f^{\prime \prime}(\eta_n)}}{{\displaystyle -\frac{b-a}{12}(\frac{h}{2})^2f^{\prime \prime}(\eta_{2n})}} \approx 4 \notag \\[2mm]
\Rightarrow \ &I - T_{2n} \approx \frac{1}{3}(T_{2n}-T_n)\ \text{\small{(误差事后估计法)}} \notag \\[2mm]
\Rightarrow \ &I \approx T_{2n} + \frac{1}{3}(T_{2n} - T_n)\ \text{\small{(误差的补偿思想)}} \notag \\[2mm]
\text{\small{同理可得:}} &I \approx S_{2n} + \frac{1}{15}(S_{2n} - S_n) \notag \\[2mm]
&I \approx C_{2n} + \frac{1}{63}(C_{2n} - C_n) \notag 
\end{align}

上式表明用$T_{2n}$作准确积分$I$的近似值时,其误差约为$\displaystyle{\frac{1}{3}(T_{2n}-T_n)}$,而实际计算常用的计算精度判断式:
$$
\left| T_{2n}-T_{n}\right| < \varepsilon\quad  \text{\small{(允许误差)}}
$$
\newpage




\item \textbf{Romberg求积法}

在复化梯形公式的步长自适应选取中,根据误差的事后估计法,利用$\displaystyle{\frac{1}{3}(T_{2n}-T_n)}$作为$T_{2n}$的补偿(误差的补偿思想),可得到比$T_{2n}$更好的结果$\tilde{T}$:
$$
\tilde{T} = \frac{4}{3}T_{2n}-\frac{1}{3}T_n
$$

经数学验证(带入展开,左右相等),有:
\begin{align}
S_n &= \frac{4}{3}T_{2n}-\frac{1}{3}T_n \notag \\[3mm]
\Rightarrow\ C_n &= \frac{16}{15}S_{2n}-\frac{1}{15}S_n \notag \\[3mm]
\Rightarrow\ R_n &= \frac{64}{63}C_{2n}-\frac{1}{64}C_n \label{Romberg}
\end{align}

其中式\ref{Romberg} 称为\textbf{Romberg公式},为Cotes公式的线性组合。

\textbf{例:}用复化梯形法及自适应选取步长计算积分
$$
I=\int_0^1 \frac{\sin x}{x}\,\mathrm{d}x\text{,取}\ \varepsilon = 10^{-7}
$$
\textbf{解:}定义$f(0)=1$,计算$f(1)=0.8414710$,则:
$$
T_2^0 = \frac{1}{2}[f(0)+f(1)] = 0.9207355
$$
将区间持续二分,直到$\left|T_2^{10}-T_2^9 \right|\leqslant \varepsilon$时,$T_2^{10}=0.9460831$,需要计算$2^{10}+1$个点函数值,而如果按照Romberg方法用线性组合计算,达到相同精度仅需$2^3+1$个点函数值,计算量大大减少,数值计算结果如下表:
\begin{table}[h]
\renewcommand\arraystretch{1.8}
\centering
\caption{数据表}
\begin{tabular}{ccccc}
\small{二分次数}$\ k$ & \small{$T_2^{k}$} & \small{$S_2^{k-1}$} & \small{$C_2^{k-2}$} & \small{$R_2^{k-3}$} \\
0 & 0.9207355 & \  & \  & \ \\
1 & 0.9397933 & 0.9461459  & \  & \ \\
2 & 0.9445135 & 0.9460869  & 0.9460830  & \ \\
3 & 0.9456909 & 0.9460833  & 0.9460831  & 0.9460831 \\
\end{tabular}
\end{table}
\end{enumerate}
\newpage

\textbf{Richardson外推加速技术}

Romberg加速计算过程的理论依据是复化梯形公式可展开成如下级数形式:
\begin{thm3}
设$f(x)\in C^\infty [a,b]$,则有:
\begin{equation}\label{th}
T(h) = I + \alpha_1 h^2 + \alpha_2 h^4 + \dots +\alpha_k h^{2k} + \dots
\end{equation}
其中,系数$\alpha_k\ (k=1,2,\dots)$与步长$h$无关。
\end{thm3}

根据式\ref{th} 可得:
\begin{equation}\label{th2}
T(\frac{h}{2}) = I + \frac{\alpha_1}{4}h^2 + \frac{\alpha_2}{16}h^4 + \frac{\alpha_3}{64}h^6 + \dots 
\end{equation}

根据式\ref{th} 和式\ref{th2} 可构造$T_1(h)$即为复化Simpson公式,并经计算可知:、:
\begin{align}
T_1(h) &= \frac{4}{3}T(\frac{h}{2}) - \frac{1}{3}T(h) = S_n\quad (Simpson)\\
T_1(h) &= I + \beta_1 h^4 + \beta_2 h^6 + \beta_3 h^8 +\dots 
\end{align}

类似地,构造复化Cotes公式$T_2(h)$:
\begin{align}
T_2(h) &= \frac{16}{15}T_1(\frac{h}{2}) - \frac{1}{15}T_1(h) = C_n \quad (Cotes)\\
T_2(h) &= I + \gamma_1 h^6 + \gamma_2 h^8 + \gamma_3 h^{10} +\dots 
\end{align}

同理,可构造Romberg公式$T_3(h)$:
\begin{align}
T_3(h) &= \frac{64}{63}T_2(\frac{h}{2}) - \frac{1}{63}T_2(h) = R_n \quad (Romberg)\\
T_3(h) &= I + \eta_1 h^8 + \eta_2 h^{10} + \eta_3 h^{12}\dots 
\end{align}

如此下去,每加速一次误差的量级便提高二阶,则Romberg求积就具有8阶收敛性.一般地,若记$T_0(h) = T(h)$,则经过$m$次$(m=1,2,\dots)$加速后:
\begin{equation}
T_m(h)=\frac{4^m}{4^m - 1}T_{m-1}(\frac{h}{2}) - \frac{1}{4^m -1}T_{m-1}(h)
\end{equation}

余项展开式为:
\begin{equation}
T_m(h)=I + \delta_1 h^{2(m+1)}+\delta_2 h^{2(m+2)} + \dots
\end{equation}
该加速过程称为\textbf{Richardson外推加速技术},其收敛性具有理论保证,实际计算时一般只加速三次($m=3$)已足够,当$m\geqslant 4$时前一个系数接近1,加速效果不明显。

\newpage

\begin{proof}
For theorem \ref{th}: 

将$f(x)$在$[x_k,x_{k+1}]$区间上的中点$x_{k+\frac{1}{2}}$处进行Taylor展开有:
$$
    f(x) = f_{x+\frac{1}{2}} + (x-x_{k+\frac{1}{2}})f_{k+\frac{1}{2}}^{\prime} + \frac{(x-x_{k+\frac{1}{2}})^2}{2!}f_{k+\frac{1}{2}}^{\prime \prime} + \frac{(x-x_{k+\frac{1}{2}})^3}{3!}f^{(3)}_{k+\frac{1}{2}} + \cdots
$$

若分别将$f(x_k)$、$f(x_{k+1})$带入Taylor展开,可得到在区间$[x_k,x_{k+1}]$上的积分:
$$
    \frac{h}{2}[f(x_k)+f(x_{k+1})] = hf_{k+\frac{1}{2}} + \frac{h}{2!} (\frac{h}{2})^2 f_{k+\frac{1}{2}}^{\prime \prime} + \frac{h}{4!}(\frac{h}{2})^4 f^{(4)}_{k+\frac{1}{2}} + \cdots
$$

进而求得:
$$
    T(h) = \frac{h}{2}\sum\limits_{k=0}\limits^{n-1}[f(x_k)+f(x_{k+1}) ] = h\sum\limits_{k=0}\limits^{n-1}f_{k+\frac{1}{2}} + \frac{h^3}{2!\times 2^2}\sum\limits_{k=0}\limits^{n-1}f_{k+\frac{1}{2}}^{\prime \prime}+\cdots
$$

若将Taylor展开式先在$[x_k,x_{k+1}]$上积分(注意奇函数积分)并求和,得:
$$
    I=\int_a^bf(x)\,\mathrm{d}x = \sum\limits_{k=0}\limits^{n-1}\int_{x_k}^{x_{k+1}}f(x)\,\mathrm{d}x = h\sum\limits_{k=0}\limits^{n-1}f_{k+\frac{1}{2}} + \frac{h^3}{3!\times 2^2}\sum\limits_{k=0}\limits^{n-1}f_{k+\frac{1}{2}}^{\prime \prime}+\cdots
$$

将$T(h)$与$I$做差可消去$h\sum\limits_{k=0}\limits^{n-1}f_{k+\frac{1}{2}}$项,得:
$$
    T(h) = I + \frac{h^3}{2!\times6}\sum\limits_{k=0}\limits^{n-1}f_{k+\frac{1}{2}}^{\prime \prime} + \frac{h^5}{4!\times 20}\sum\limits_{k=0}\limits^{n-1}f_{k+\frac{1}{2}}^{(4)} + \cdots
$$

若用$f^{\prime \prime}(x)$代替$f(x)$进行Taylor展开,并在区间$[a,b]$上积分,得:
$$
    f^{\prime}(b) - f^{\prime}(a) = h\sum\limits_{k=0}\limits^{n-1}f_{k+\frac{1}{2}}^{\prime \prime} + \frac{h^3}{3!\times 2^2}\sum\limits_{k=0}\limits^{n-1}f_{k+\frac{1}{2}}^{(4)}+ \frac{h^5}{5!\times 2^4}\sum\limits_{k=0}\limits^{n-1}f_{k+\frac{1}{2}}^{(6)} + \cdots
$$

利用该式可消去$T(h)$与$I$关系式中的$h\sum\limits_{k=0}\limits^{n-1}f_{k+\frac{1}{2}}^{\prime \prime}$项,得:
$$
T(h) = I + \frac{h^2}{2!\times6}[f^{\prime}(b) - f^{\prime}(a)] - \frac{h^5}{4!\times 30}\sum\limits_{k=0}\limits^{n-1}f_{k+\frac{1}{2}}^{(4)} +\cdots
$$

重复上述过程,可分别消去$h\sum\limits_{k=0}\limits^{n-1}f_{k+\frac{1}{2}}^{(4)}$、$h\sum\limits_{k=0}\limits^{n-1}f_{k+\frac{1}{2}}^{(6)}$等项,即可得到式 \ref{th}.
\end{proof}


\newpage
\subsection{Other Simple Integral Formula}
\begin{enumerate}
\item \textbf{左矩形公式}
$$
I_L = (b-a)f(a)
$$
\item \textbf{中矩形公式}
$$
I_M = (b-a)f(\frac{a+b}{2})
$$
\item \textbf{右矩形公式}
$$
I_R = (b-a)f(b)
$$
\end{enumerate}

对应余项:将$f(x)$分别在$x=a$、$x=b$、$\displaystyle{x=\frac{a+b}{2}}$处用Taylor展开:
\begin{align}
f(x) &= f(x_0) + f^{\prime}(x_0)(x-x_0) + \frac{f^{\prime \prime}}{2}(x_0)(x-x_0)^2 + \dots \notag \\[4mm]
\int_a^bf(x)\,\mathrm{d}x &\approx (b-a)f(a) + f^{\prime}(a) \int_a^b(x-a)\,\mathrm{d}x = (b-a)f(a) + \frac{f^{\prime}(\xi)}{2}(b-a)^2 \ \xi \in (a,b)\notag \\[4mm]
\Rightarrow R_L &= \frac{f^{\prime}(\xi)}{2}(b-a)^2 \ \xi \in (a,b)\notag \\[4mm]
\int_a^bf(x)\,\mathrm{d}x &\approx (b-a)f(b) + f^{\prime}(b) \int_a^b(x-b)\,\mathrm{d}x = (b-a)f(a) - \frac{f^{\prime}(\xi)}{2}(b-a)^2 \ \xi \in (a,b)\notag \\[4mm]
\Rightarrow R_R&= - \frac{f^{\prime}(\xi)}{2}(b-a)^2 \ \xi \in (a,b)\notag \\[4mm]
\int_a^bf(x)\,\mathrm{d}x &\approx (b-a)f(\frac{a+b}{2}) + f^{\prime}(\frac{a+b}{2}) \int_a^b(x-\frac{a+b}{2})\,\mathrm{d}x + \frac{f^{\prime \prime}(\frac{a+b}{2})}{2}\int_a^b (x-\frac{a+b}{2})^2 \,\mathrm{d}x\notag \\[4mm]
&= (b-a)f(\frac{a+b}{2}) + \frac{f^{\prime \prime}(\xi)}{24}(b-a)^3\  \xi \in (a,b)\notag \\[4mm]
\Rightarrow R_M &= \frac{f^{\prime \prime}(\xi)}{24}(b-a)^3 \ \xi \in (a,b)\notag
\end{align}


\newpage
\section{Gauss型数值积分公式}
考虑带权积分,并构造插值型求积公式:
\begin{align}
I &= \int_a^b \omega(x)f(x)\,\mathrm{d}x \approx \sum\limits_{k=0}\limits^n A_k f(x_k) = \sum\limits_{k=0}\limits^n [\int_a^b\omega(x) l_k(x)\,\mathrm{d}x] f(x_k)
\\[3mm]
R &= \int_a^b \omega(x)f(x)\,\mathrm{d}x-\sum\limits_{k=0}\limits^n A_k f(x_k) = \int_a^b \omega(x) \frac{f^{(n+1)}(\xi)}{(n+1)!}\omega_{n+1}(x)\,\mathrm{d}x
\end{align}\label{gauss}

显然,由余项可知求积公式对$n+1$个点至少具有$n$次代数精度。如果将求积节点和求积系数作为代数精度,则共有$2n+2$个未知数,恰当地选择这些参数可以使得公式\ref{gauss} 对$f(x)=1,x,x^2,\dots,x^{2n+1}$精确成立,即具有$2n+1$次代数精度。

\textbf{定义:}若求积公式\ref{gauss} 具有$2n+1$次代数精度,则称公式\ref{gauss} 为\textbf{Gauss型求积公式},对应的求积节点$x_k\ (k=0,1,\dots,n)$称为Gauss点,求积系数$A_k$称为Gauss系数。

\textbf{Eg.} 对$\int_{-1}^{1} f(x)\,\mathrm{d}x \approx A_0 f(x_0) + A_1 f(x_1)$,确定其求积系数$A_0, A_1$与求积节点$x_0, x_1$,使其具有最高的代数精度:
{\setlength\abovedisplayskip{0cm} \setlength\belowdisplayskip{0.2cm}
\begin{align}
    f(x) &= 1,\quad 2 = A_0 + A_1 \notag \\
    f(x) &= x,\quad 0 = A_0 x_0 + A_1 x_1 \notag \\ 
    f(x) &= x^2,\  \frac{2}{3} = A_0 x_0^2 + A_1 x_1^2 \notag \\ 
    f(x) &= x^3,\ \  0 = A_0 x_0^3 + A_1 x_1^3 \notag 
\end{align}}

从上式联立可解出四个待定参数,并且可以证明$f(x)=x^4$时不成立,从而该积分公式至少具有3次代数精度:

\textbf{Gauss型求积公式一定是插值型求积公式,具有最高的代数精度,且当$f(x)\in C[a,b]$时总是稳定和收敛的。若$f(x)\in C^{2n+2}[a,b]$,则Gauss型求积公式的余项为:}
\begin{equation}
    R = \int_a^b \omega(x)f(x)\,\mathrm{d}x - \sum\limits_{k=0}\limits^n A_k f(x_k) = \frac{f^{(2n+2)}(\xi)}{(2n+2)!}\int_a^b \omega(x)\omega_{n+1}^2(x)\,\mathrm{d}x\quad \xi \in (a,b)
\end{equation}

余项的证明需要构造$2n+1$次标准Hermite插值多项式$H_{2n+1}(x)$,因为Gauss公式对其精确成立,有:
$$
    R = \int_a^b \omega(x)[f(x) - H_{2n+1}(x)]\,\mathrm{d}x = \int_a^b \omega(x) \frac{f^{(2n+2)}(\eta)}{(2n+2)!}\omega^2_{n+1}(x)\,\mathrm{d}x
$$

\newpage
\begin{thm3}
    对于插值型求积公式的求积节点$x_k\ (k=0,1,\dots,n)$是Gauss点的充要条件是在$[a,b]$上,以这些点为零点的$n+1$次多项式$\omega_{n+1}(x)$与任意次数不超过$n$的多项式$p(x)$关于权函数$\omega(x)$正交,即
    \begin{equation}
        \int_a^b \omega(x)\omega_{n+1}(x)p(x)\,\mathrm{d}x = 0
    \end{equation}
\end{thm3}
\begin{proof}
    证明过程如下:
    \begin{enumerate}
        \item 必要性:取$f(x)=\omega_{n+1}(x)p(x)$,其次数$\leqslant 2n+1$,若节点为Gauss点,则其对$2n+1$次的多项式精确成立,有:
            $$
                \int_a^b \omega(x)\omega_{n+1}(x)p(x)\,\mathrm{d}x = \sum\limits_{k=0}\limits^{n} A_k \omega_{n+1}(x_k)p(x_k) = 0
            $$
        \item 充分性:对任意次数$\leqslant 2n+1$的多项式$f(x)$,构造次数$\leqslant n$的多项式$p(x),\ q(x)$满足:
            $$
                f(x) = p(x)\omega_{n+1}(x) + q(x)
            $$
            则有:
            \begin{align*}
                \int_a^b \omega(x)f(x)\,\mathrm{d}x &= \int_a^b \omega(x)p(x)\omega_{n+1}(x)\,\mathrm{d}x + \int_a^b \omega(x)q(x)\,\mathrm{d}x \\
                &= \int_a^b \omega(x)q(x)\,\mathrm{d}x = \sum\limits_{k=0}\limits^{n}A_k q(x_k) \quad \text{\small (插值型至少具有n次代数精度)}\\
                &= \sum\limits_{k=0}\limits^{n}A_k f(x_k)
            \end{align*}
            因此对任意$2n+1$次多项式成立,即该求积节点为Gauss点。
    \end{enumerate}
\end{proof}

\noindent Gauss型求积公式的稳定性证明:
\begin{proof}
    类似Newton-Cotes公式的证明,只需证明其系数全为正即可。由于Gauss公式具有$2n+1$次代数精度,因此取$f(x)=l_k^2(x)$时,Gauss公式精确成立,有:
    \begin{equation*}
        \int_a^b \omega(x)l_k^2(x)\,\mathrm{d}x = \sum\limits_{k=0}\limits^{n} A_i l_k^2(x_i) = A_k > 0
    \end{equation*}
    所以Gauss型求积公式总是稳定的
\end{proof}


\newpage
\subsection*{常见的Gauss型求积公式}
\begin{enumerate}
    \item \textbf{Gauss-Legendre求积公式}\\
        积分区间$[a,b]=[-1,1],\ \omega(x)=1$,Gauss点$x_k$为$n+1$次Legendre多项式$p_{n+1}(x)$的零点:
        \begin{align*}
            p_{n+1}(x) &= \frac{1}{(n+1)!2^{n+1}}\frac{\,\mathrm{d}^{n+1}}{\,\mathrm{d}x^{n+1}}(x^2-1)^{n+1} \\[3mm]
            A_k &= \int_{-1}^1 l_k(x)\,\mathrm{d}x = \frac{2}{(1-x_k^2)[p_{n+1}^{\prime}(x_k)]^2} \quad (k=0,1,\dots,n) \\[3mm]
            R &= \frac{2^{2n+3}[(n+1)!]^4}{(2n+3)[(2n+2)!]^3}f^{(2n+2)}(\xi) \quad \xi \in (-1,1)
        \end{align*}
    \item \textbf{Gauss-切比雪夫求积公式}\\
        积分区间$[a,b]=[-1,1],\ \omega(x)=(1-x^2)^{-\frac{1}{2}}$,Gauss点$x_k$为$n+1$次切比雪夫多项式的零点:
        \begin{align*}
            & \int_{-1}^1 \frac{1}{\sqrt{1-x^2}}f(x)\,\mathrm{d}x \approx \sum\limits_{k=0}\limits^n A_k f(x_k) \\[3mm]
            & A_k = \frac{\pi}{n+1},\quad x_k = \cos(\frac{2k+1}{2n+2}\pi) \quad (k=0,1,\dots,n) \\[3mm]
            & R = \frac{\pi}{2^{2n+1}(2n+2)!}f^{(2n+2)}(\xi) \quad \xi \in (-1,1)
        \end{align*}
    \item \textbf{Gauss-Laguerre求积公式}\\
        积分区间$[a,b]=[0,+\infty),\ \omega(x)=\mathrm{e}^{-x}$,Gauss点$x_k$为$n+1$次Laguerre多项式$L_{n+1}(x)$的零点:
        \begin{align*}
            & \int_0^{+\infty} \mathrm{e}^{-x}f(x)\,\mathrm{d}x \approx \sum\limits_{k=0}\limits^{n}A_k f(x_k) \\[3mm]
            & L_{n+1}(x) = \mathrm{e}^x \frac{\,\mathrm{d}^{n+1}}{\,\mathrm{d}x^{n+1}}(x^{n+1}\mathrm{e}^{-x}) \\[3mm]
            & A_k = \frac{[(n+1)!]^2}{x_k[L_{n+1}^{\prime}(x_k)]^2} \quad (k=0,1,\dots,n) \\[3mm]
            & R = \frac{[(n+1)!]^2}{(2n+2)!}f^{(2n+2)}(\xi) \quad \xi \in [0,+\infty)
        \end{align*}
    \item \textbf{Gauss-Hermite求积公式}\\
        积分区间$(-\infty, +\infty),\ \omega(x)=\mathrm{e}^{-x^2}$,Gauss点$x_k$为$n+1$次Hermite多项式$H_{n+1}(x)$的零点:
        \begin{align*}
            & \int_{-\infty}^{+\infty} \mathrm{e}^{-x^2}f(x)\,\mathrm{d}x \approx \sum\limits_{k=0}\limits^{n}A_k f(x_k) \\[3mm]
            & H_{n+1}(x) = (-1)^{n+1}\mathrm{e}^{x^2}=\frac{\,\mathrm{d}^{n+1}}{\,\mathrm{d}x^{n+1}}(e^{-x^2}) \\[3mm]
            & A_k = \frac{2^{n+2} (n+1)!}{[H_{n+1}^{\prime}(x_k)]^2}\sqrt{\pi} \quad (k=0,1,\dots,n) \\[3mm]
            & R = \frac{(n+1)!\sqrt{\pi}}{2^{n+1}(2n+2)!}f^{(2n+2)}(\xi) \quad \xi \in (-\infty,+\infty)
        \end{align*}
\end{enumerate}


\newpage
\section{数值微分}
数值微分即研究通过已知函数值来表示函数导数的方法。

\subsection{Taylor级数展开法}
$f^{\prime}(x_0)$可以通过Taylor级数展开获得类似差商的近似估计计算公式:
\begin{align*}
    f(x_0+h) &= f(x_0) + hf^{\prime}(x_0) + \frac{h^2}{2!}f^{\prime \prime}(\xi) \\[3mm]
    \Rightarrow \  f^{\prime}(x_0) &= \frac{f(x_0+h) - f(x_0)}{h} + O(h) \\[3mm]
    \Rightarrow \  f^{\prime}(x_0) &\approx \frac{f(x_0+h) - f(x_0)}{h} \\[3mm]
    f^{\prime}(x_0) &\approx \frac{f(x_0) - f(x_0-h)}{h} \\[3mm]
    f^{\prime}(x_0) &\approx \frac{f(x_0+h) - f(x_0-h)}{2h}
\end{align*}

该方法并不是$h$越小越好,当$h$特别小时,容易导致分子上两个很接近的数相减,进而造成有效数字的严重损失。为了克服Taylor展开的缺陷,利用外推方法来提高精度,根据Taylor展开有:
\begin{align*}
    & f(x_0 \pm h) = f(x_0) \pm hf^{\prime}(x_0) + \frac{h^2}{2!}f^{\prime \prime}(x_0) \pm \frac{h^3}{3!}f^{\prime \prime \prime}(x_0) + \frac{h^4}{4!}f^{(4)}(x_0) \pm \cdots \\[4mm]
    \Rightarrow \ & G(h) = \frac{f(x_0+h)-f(x_0-h)}{2h} = f^{\prime}(x_0) + [\frac{h^2}{3!}f^{(3)}(x_0) + \frac{h^4}{5!}f^{(5)}(x_0) + \frac{h^6}{7!}f^{(7)}(x_0) + \cdots]
\end{align*}

类似数值积分的Richardson外推,可建立如下外推方法:
\begin{equation}
    \begin{cases}
        G_1(h) = G(h) \\[4mm]
        G_{m+1}(h) = {\displaystyle \frac{4^mG_m(\frac{h}{2}) - G_m(h)}{4^m - 1}} \qquad m = 1,2,\dots\\
    \end{cases}
\end{equation}

外推$m$次之后$f^{\prime}(x)$的余项:
\begin{equation*}
    f^{\prime}(x_0) - G_{m+1}(h) = O(h^{2(m+1)})
\end{equation*}


\newpage
\subsection{数值微分的隐格式}
微分是积分的逆运算,从而可将数值微分问题转化为数值积分问题,如对区间$[a,b]$进行$n$等分,节点${\displaystyle x_k=a+kh\ (k=1,2,\dots,n),\ h=\frac{b-a}{n}}$,由于:
\begin{equation*}
    f(x_{k+1}) - f(x_{k-1}) = \int_{x_{k-1}}^{x_{k+1}} f^{\prime}(x)\,\mathrm{d}x \quad (k=1,2,\dots,n-1)
\end{equation*}

对右端进行Simpson积分:
\begin{equation*}
    f(x_{k+1}) - f(x_{k-1}) \approx \frac{h}{3}[f^{\prime}(x_{k-1}) + 4f^{\prime}(x_k) + f^{\prime}(x_{k+1})] \quad (k=1,2,\dots,n-1)
\end{equation*}

记$m_k \approx f^{\prime}(x_k)$,给定边界$m_0 = f^{\prime}(x_0),\ m_n = f^{\prime}(x_n)$,则有含$n-1$个未知数的$n-1$个方程组成的方程组,其系数矩阵严格对角占优,因而存在唯一解。此方程组称为\textbf{数值微分的隐格式},如下:
\begin{equation}
    \begin{cases}\displaystyle
        \ m_{k-1} + 4m_k + m_{k+1} = \frac{3}{h} [f(x_{k+1}) - f(x_{k-1})] \quad (k=1,2,\dots,n-1)\\[3mm]
        \ m_0 = f^{\prime}(x_0) \\[3mm]
        \ m_n = f^{\prime}(x_n) \\
    \end{cases}
\end{equation}


\subsection{插值型求导公式}
\begin{enumerate}
    \item \textbf{基于Lagrange插值的方法}\\
        构造Lagrange插值并求一阶导数:
        \begin{align*}
            f(x) &= L_n(x) + \frac{f^{(n+1)}(\xi)}{(n+1)!}\omega_{n+1}(x) \quad \xi \in (x_0,x_n) \\[3mm]
            \Rightarrow f^{\prime}(x) &= L_n^{\prime}(x) +  \frac{f^{(n+1)}(\xi)}{(n+1)!}\omega_{n+1}(x)^{\prime} +  \frac{\omega_{n+1}(x)}{(n+1)!}\frac{\,\mathrm{d}}{\,\mathrm{d}x}f^{(n+1)}(\xi)
        \end{align*}
        由于对$f^{(n+1)(\xi)}$的导数无法作估计,利用节点处$\omega_{n+1}(x_k)=0$限定求$x_k$处的导数:
        \begin{equation}
            f^{\prime}(x_k) - L_n^{\prime}(x_k) = \frac{f^{(n+1)}(\xi)}{(n+1)!}\omega_{n+1}(x_k)^{\prime}
        \end{equation}
        若$\left| f^{\prime}(x) \right| \leqslant M,\ x\in(a,b)$,则:
        \begin{equation*}
            \left| f^{\prime}(x_k) - L_n^{\prime}(x_k) \right| \leqslant M \frac{(b-a)^n}{(n+1)!} \overset{n\rightarrow \infty}{\longrightarrow} 0 \quad \Rightarrow \ f^{\prime}(x_k) \approx L_n^{\prime}(x_k)
        \end{equation*}
        \newpage

        \textbf{基于Lagrange插值方法的等距节点处常用数值微分公式:}
        \begin{enumerate}
            \item 一阶两点公式($n=1$)
                \begin{align}
                    f^{\prime}(x_0) &= \frac{1}{h}[f(x_1)-f(x_0)] - \frac{h}{2}f^{\prime \prime}(\xi_1) \quad \xi_1 \in (x_0,x_1) \\[3mm]
                    f^{\prime}(x_1) &= \frac{1}{h}[f(x_1)-f(x_0)] + \frac{h}{2}f^{\prime \prime}(\xi_2) \quad \xi_2 \in (x_0,x_1)
                \end{align}
            \item 一阶三点公式($n=2$)
            \begin{align}
                f^{\prime}(x_0) &= \frac{1}{2h}[-3f(x_0)+4f(x_1)-f(x_2)] + \frac{h^2}{3}f^{(3)}(\xi_1) \quad &\xi_1 \in (x_0,x_2) \\[3mm]
                f^{\prime}(x_1) &= \frac{1}{2h}[-f(x_0)+f(x_2)] - \frac{h^2}{6}f^{(3)}(\xi_2) \quad &\xi_2 \in (x_0,x_2) \\[3mm]
                f^{\prime}(x_2) &= \frac{1}{2h}[f(x_0)-4f(x_1)+3f(x_2)] + \frac{h^2}{3}f^{(3)}(\xi_3) \quad &\xi_3 \in (x_0,x_2)
            \end{align}
            \item 二阶三点公式($n=2$),其中$\xi_i \in (x_0,x_2)\ (i=1,2,\dots,5)$.
            \begin{align}
                f^{\prime}(x_0) &= \frac{1}{h^2}[f(x_0)-2f(x_1)+f(x_2)] -hf^{(3)}(\xi_1) + \frac{h^2}{6}f^{(4)}(\xi_2) \\[3mm]
                f^{\prime}(x_1) &= \frac{1}{h^2}[f(x_0)-2f(x_1)+f(x_2)] - \frac{h^2}{12}f^{(4)}(\xi_3) \\[3mm]
                f^{\prime}(x_2) &= \frac{1}{h^2}[f(x_0)-2f(x_1)+f(x_2)] + hf^{(3)}(\xi_4) - \frac{h^2}{6}f^{(4)}(\xi_5) 
            \end{align}
        \end{enumerate}
    \item \textbf{基于三次样条插值的方法}\\
        由估计式$\left|\right| f^{(k)}(x)-s^{(k)}(x) \left| \right|_{\infty} \leqslant C_k h^{4-k}\left|\right| f^{(4)}(x)\left|\right|_{\infty}\ (k=0,1,2,3)$有:
        \begin{equation}
            f^{(k)}(x) \approx s^{(k)}(x) \quad (k=0,1,2,3)
        \end{equation}
        在等距节点条件下,记$m_k=S_{\prime}(x_k)$,并给定边界条件$m_0=f^{\prime}(x_0),\ x_n=f^{\prime}(x_n)$,则解如下方程组可得到$m_k$作为$f^{\prime}(x_k)$的近似值:
        \begin{equation}
            m_{k-1} + 4m_k + m_{k+1} = \frac{3}{h} (y_{k+1} - y_{k-1})
        \end{equation}
\end{enumerate}

基于Lagrange插值的方法只能求节点处的近似微分值,而基于三次样条插值的方法可以用来求插值范围内任意$x$(包含节点)处的近似微分。


% \end{document}

