% \documentclass[a4paper, 11pt, oneside, onecolumn, titlepage]{ctexrep}

% \usepackage{makeidx,amsmath}
% \usepackage{unicode-math}
% \usepackage{setspace}
% \usepackage{amsthm}
% \usepackage{geometry}
% \usepackage{fancyhdr}
% \usepackage{tikz}

% \geometry{left=25mm,right=25mm,top=30mm,bottom=30mm}
% \setmathfont{XITS Math}

% \linespread{1.5}
% \title{Numerical Integration \& Numerical Differentiation}
% \author{Brassicaaa}
% \date{\today}

% \setlength{\headheight}{27pt}

% \begin{document}
% \pagestyle{fancy}
% \lhead{Notes for Numerical Analysis}
% \setcounter{page}{1}
% \pagenumbering{arabic}

\chapter{解线性方程组-直接法}

\section{Gauss消去法}

对线性代数方程组:
\begin{equation*}
    \boldsymbol{Ax} = \boldsymbol{b}
\end{equation*}

其中,系数矩阵$\boldsymbol{A} = (a_{ij})_{n\times n}$非奇异,$\boldsymbol{x}=(x_1,x_2,\dots,x_n)^{T}$,右端项$\boldsymbol{b}=(b_1,b_2,\dots,b_n)^T$。

\textbf{Gauss消去法}的基本思想是将方程组转化为上三角形方程组,然后进行求解,包括\textbf{消元}和\textbf{回代}两大过程:
\begin{enumerate}
    \item \textbf{消元过程}:记初始方程组为$\boldsymbol{A}^{(1)}\boldsymbol{x} = \boldsymbol{b}^{(1)}$,用第一行方程与剩下每行方程分别做线性运算消去第$2$至$n$行的方程中的$x_1$项,得到$\boldsymbol{A}^{(2)}\boldsymbol{x}=\boldsymbol{b}^{(2)}$:
    \begin{align*}
        a_{11}^{(1)}x_1 + a_{12}^{(1)}x_2 + \dots + a_{1n}^{(1)} &= b_1^{(1)} \\
        a_{22}^{(2)}x_2 + \dots + a_{2n}^{(2)}x_n &= b_2^{(2)} \\
        \cdots \cdots & \cdots \cdots \\
        a_{n2}^{(2)}x_2 + \dots + a_{nn}^{(2)}x_n &= b_n^{(2)}
    \end{align*}
    然后用第二行与剩下$3$至$n$行分别做线性运算消去方程中的$x_2$项,得到$\boldsymbol{A}^{(3)}\boldsymbol{x}=\boldsymbol{b}^{(3)}$;
    以此类推,完成第$n-1$步后得到上三角矩阵$\boldsymbol{A}^{(n)}\boldsymbol{x}=\boldsymbol{b}^{(n)}$:
    \begin{align*}
        a_{11}^{(1)}x_1 + a_{12}^{(1)}x_2 + a_{13}^{(1)}x_3 + \dots + a_{1n}^{(1)} &= b_1^{(1)} \\
        a_{22}^{(2)}x_2 + a_{23}^{(2)}x_3 + \dots + a_{2n}^{(2)}x_n &= b_2^{(2)} \\
        a_{33}^{(3)}x_3 + \dots + a_{3n}^{(3)}x_n &= b_2^{(3)} \\
        \cdots \cdots & \cdots \cdots \\
        a_{nn}^{(2)}x_n &= b_n^{(n)}
    \end{align*}
    该过程的求解公式如下:
    \begin{equation}
        \begin{cases}
            \ l_{ik} &= a_{ik}^{(k)} / a_{kk}^{(k)} \quad (i=k+1,k+2,\dots,n) \\[3mm]
            \ a_{ij}^{(k+1)} &= a_{ij}^{(k)} - l_{ik}a_{kj}^{(k)} \quad (i,j = k+1,k+2,\dots,n) \\[3mm]
            \ b_i^{(k+1)} &= b_i^{(k)} - l_{ik}b_k^{(k)} \quad (i=k+1,k+2,\dots,n)
        \end{cases}
    \end{equation}

    \item \textbf{回代过程}:当$a_{nn}^{(n)}\neq 0$时,从下往上代入方程求解即可,该上三角方程组的求解公式如下:
    \begin{equation}
        \begin{cases}
            \ x_n = b_n^{(n)} / a_{nn}^{(n)} \\[3mm]
            \ x_k = [b_k^{(k)} - \sum\limits_{j=k+1}^{n} a_{kj}^{(k)} x_j] / a_{kk}^{(k)} \quad (k=n-1,n-2,\dots,1)
        \end{cases}
    \end{equation}
\end{enumerate}

Gauss消去法的运算量比Gramer法则解方程组小得多,实现这一过程的充要条件是主元素$a_{ii}^{(i)} \neq 0\ (i=1,2,\dots,n-1)$。这一条件与原方程组的系数矩阵$\boldsymbol{A}$的联系有如下定理:
\newtheorem{th4}{Theorem}[chapter]
\begin{th4}
    主元素$a_{ii}^{(i)} \neq 0\ (i=1,2,\dots,n-1)$的充要条件是矩$\boldsymbol{A}$的顺序主子式$\boldsymbol{D}_i \neq 0\ (i=1,2,\dots,k;\ k\leqslant 0)$.
\end{th4}

该定理可用数学归纳法证明,其中顺序主子式$\boldsymbol{D}_k$的定义:
\begin{equation}
    \boldsymbol{D}_k = 
        \begin{vmatrix}
            a_{11}^{(1)} & \cdots & a_{1k}^{(1)} \\
            \  & \ddots & \vdots \\
            \  & \  & a_{kk}^{(k)}
        \end{vmatrix}
        = a_{11}^{(1)}a_{22}^{(2)}\cdots a_{kk}^{(k)}
\end{equation}

并有如下\textbf{推论}:

若$\boldsymbol{A}$的顺序主子式$\boldsymbol{D}_i \neq 0\ (i=1,2,\dots,n-1)$,则:
\begin{align*}
    a_{11}^{(1)} &= \boldsymbol{D}_1 \\[4mm]
    a_{kk}^{(k)} &= \frac{\boldsymbol{D}_k}{\boldsymbol{D}_{k-1}} \quad (k=2,3,\dots,n)
\end{align*}

所以,Gauss消元过程完成$\Leftrightarrow\ a_{ii}^{(i)}\neq 0\ \Leftrightarrow D_i\neq 0\quad (i=1,2,\dots,n)$.


\newpage
\section{主元素消元法}
Gauss消元过程中,要求主元$a_{ii}^{(i)}\neq 0\ (i=1,2,\dots,n)$,当主元相对很小时,以它为除数会导致其他元素数量级的巨大增长,舍入误差就会扩散,导致计算结果严重失真。主元素消元法针对此问题进行改进,全部或部分地选取绝对值最大的元素为主元素,使之成为一种有效(稳定)的算法。

\subsection{全主元素消元法}
在每一步消元进行前,将系数矩阵中绝对值最大的元素通过交换行、列移动到消元行的首位,再进行消元。如在第一步消元进行前,在整个矩阵$\boldsymbol{A}$中选取绝对值最大的元素作为主元:
\begin{equation*}
    \left| a_{i_1,j_1} \right| = \max\limits_{1\leqslant i,j\leqslant n} \left| a_{ij} \right| \neq 0
\end{equation*}
将$a_{i_1,j_1}$换到第一行第一列,然后进行消元;第二步消元前将除第一行外的所有元素中绝对值最大的元素交换到第二行第二列位置,再进行消元,重复至$n-1$步得:
\begin{equation*}
    \begin{bmatrix}
        a_{11} & a_{12} & \cdots & a_{1n} \\
        \      & a_{21} & \cdots & a_{2n} \\
        \      & \      & \ddots & \vdots \\
        \      & \      & \      & a_{nn} \\
    \end{bmatrix}\ 
    \begin{bmatrix}
        y_1 \\ y_2 \\ \vdots \\ y_n \\
    \end{bmatrix} = 
    \begin{bmatrix}
        b_1 \\ b_2 \\ \vdots \\ b_n \\
    \end{bmatrix}
\end{equation*}
从而避免了舍入误差。这里的$y_1,y_2,\dots,y_n$为$x_1,x_2,\dots,x_n$的某一个排列,因为进行了列交换。

\subsection{列主元素消元法}
列主元素消元法只在每一列选取绝对值最大的元素作为主元(只交换行),减小了计算量也能保证同样的效果,并且由于没有列的交换,所以$x_1,x_2,\dots,x_n$的顺序也没有发生变化:
\begin{equation*}
    \begin{bmatrix}
        a_{11} & a_{12} & \cdots & a_{1n} \\
        \      & a_{21} & \cdots & a_{2n} \\
        \      & \      & \ddots & \vdots \\
        \      & \      & \      & a_{nn} \\
    \end{bmatrix}\ 
    \begin{bmatrix}
        x_1 \\ x_2 \\ \vdots \\ x_n \\
    \end{bmatrix} = 
    \begin{bmatrix}
        b_1 \\ b_2 \\ \vdots \\ b_n \\
    \end{bmatrix}
\end{equation*}



\newpage
\section{矩阵三角解法}

\subsection{Doolittle分解(LU分解)}
Gauss消元法的过程完成了矩阵$\boldsymbol{A}$的一个三角分解:
\begin{equation}
    \boldsymbol{A} = \boldsymbol{LU}
\end{equation}
其中$\boldsymbol{L}$为下三角矩阵,$\boldsymbol{U}$为上三角矩阵,若$n$阶矩阵$A$的顺序主子式$D_i \neq 0\ (i=1,2,\dots,n-1)$,则该分解唯一。若$\boldsymbol{A}$非奇异,则$\boldsymbol{U}$也非奇异。其详细过程如下:

消元的第一步等同于将系数矩阵$\boldsymbol{A}$($\boldsymbol{A}^{(1)}$)和$\boldsymbol{b}$($\boldsymbol{b}^{(1)}$)左乘一个初等变换矩阵:
\begin{equation*}
    \boldsymbol{A}^{(2)} = \boldsymbol{L}_1^{-1} \boldsymbol{A}^{(1)},\quad \boldsymbol{b}^{(2)} = \boldsymbol{L}_1^{-1} \boldsymbol{b}^{(1)},\quad L_1^{-1} = \begin{bmatrix}
        1 & \  & \ & \ & \  \\
        -l_{21} & 1 & \ & \ & \  \\
        -l_{31} & 0 & 1 & \ & \  \\
        \vdots & \vdots & \vdots & \ddots & \  \\
        -l_{n1} & 0 & 0 & \cdots & 1 \\
    \end{bmatrix},\quad l_{i1} = \frac{a_{i1}^{(1)}}{a_{11}^{(1)}}\ (i=2,3,\dots,n)
\end{equation*}

如此进行$n-1$步后得到:
\begin{equation*}
    \begin{cases}
        \ \boldsymbol{L}_{n-1}^{-1} \boldsymbol{L}_{n-2}^{-1} \cdots \boldsymbol{L}_{2}^{-1} \boldsymbol{L}_{1}^{-1} \boldsymbol{A}^{(1)} = \boldsymbol{A}^{(n)} = \boldsymbol{U} \\
        \ \boldsymbol{L}_{n-1}^{-1} \boldsymbol{L}_{n-2}^{-1} \cdots \boldsymbol{L}_{2}^{-1} \boldsymbol{L}_{1}^{-1} \boldsymbol{b}^{(1)} = \boldsymbol{b}^{(n)} \\
    \end{cases}
\end{equation*}

记$\boldsymbol{L} = \boldsymbol{L}_1 \boldsymbol{L}_2 \dots \boldsymbol{L}_n$,则有$\boldsymbol{LU} = \boldsymbol{A^{(1)}} = \boldsymbol{A}$,具体形式为:
\begin{equation*}
    \boldsymbol{L} = \begin{bmatrix}
        1 & \ & \ & \ & \ \\
        l_{21} & 1 & \ & \ & \ \\
        l_{31} & l_{32} & 1 & \ & \ \\
        \vdots & \vdots & \ & \ddots & \ \\
        l_{n1} & l_{n2} & \cdots & l_{n,n-1} & 1 \\ 
    \end{bmatrix},\quad \boldsymbol{U} = 
    \begin{bmatrix}
        u_{11} & u_{12} & u_{13} & \cdots & u_{1n} \\
        \      & u_{22} & u_{23} & \cdots & u_{2n} \\
        \      & \      & u_{33} & \cdots & u_{3n} \\
        \      & \      & \      & \ddots & \vdots \\
        \      & \      & \      & \      & u_{nn} \\
    \end{bmatrix}
\end{equation*}

则$\boldsymbol{Ax} = \boldsymbol{LUx} = \boldsymbol{b}$可转化为两个三角形方程求解(行列交替进行):
\begin{equation}
    \begin{cases}
        \ \boldsymbol{Ly} = \boldsymbol{b} \\
        \ \boldsymbol{Ux} = \boldsymbol{y} \\
    \end{cases},\quad
    \begin{cases}
        \ y_1 = b_1 \\
        \ y_i = b_i - \sum\limits_{j=1}\limits^{i-1}l_{ij}y_j
    \end{cases},\quad
    \begin{cases}
        \ x_n = y_n / u_{nn} \\
        \ x_i = (y_i - \sum\limits_{j=i+1}\limits^{n} u_{ij}x_j)
    \end{cases}
\end{equation}


\newpage
\textbf{补充:}$\boldsymbol{LU}$\textbf{分解具有唯一性}$\Leftrightarrow D_i\neq 0\ (i=1,2,\dots,n-1)$

\textbf{Eg.1}
\begin{equation*}
    \boldsymbol{A} = \begin{bmatrix}
        1 & 2 & 3 \\
        2 & 4 & 4 \\
        3 & 5 & 6 \\
    \end{bmatrix}\quad
    \Rightarrow \boldsymbol{D}_1 = 1,\ \boldsymbol{D}_2 = 0
\end{equation*}

将$\boldsymbol{A}$消元一次:
\begin{equation*}
    \boldsymbol{A}^{(2)} = \begin{bmatrix}
        1 & 2 & 3 \\
        0 & 0 & -2 \\
        0 & -1 & -3 \\
    \end{bmatrix}
\end{equation*}

其$\boldsymbol{LU}$分解不存在。


\textbf{Eg.2}
\begin{equation*}
    \boldsymbol{A} = \begin{bmatrix}
        1 & 2 & 3 \\
        2 & 4 & 4 \\
        3 & 6 & 6 \\
    \end{bmatrix}\quad
    \Rightarrow \boldsymbol{D}_1 = 1,\ \boldsymbol{D}_2 = 0
\end{equation*}

将$\boldsymbol{A}$消元一次:
\begin{equation*}
    \boldsymbol{A}^{(2)} = \begin{bmatrix}
        1 & 2 & 3 \\
        0 & 0 & -2 \\
        0 & 0 & -3 \\
    \end{bmatrix} = \boldsymbol{U}
\end{equation*}

其$\boldsymbol{LU}$分解存在但不唯一。


\textbf{Eg.3}
\begin{equation*}
    \boldsymbol{A} = \begin{bmatrix}
        1 & 2 & 3 \\
        2 & 3 & 4 \\
        3 & 4 & 6 \\
    \end{bmatrix}\quad
    \Rightarrow \boldsymbol{D}_1 = 1,\ \boldsymbol{D}_2 = -1
\end{equation*}

将$\boldsymbol{A}$消元:
\begin{equation*}
    \boldsymbol{A} \overset{\text{消元1次}}{\longrightarrow} \begin{bmatrix}
        1 & 2 & 3 \\
        0 & -1 & -2 \\
        0 & -2 & -3 \\
    \end{bmatrix} \overset{\text{消元2次}}{\longrightarrow}
    \begin{bmatrix}
        1 & 2 & 3 \\
        0 & -1 & -2 \\
        0 & 0 & 1 \\
    \end{bmatrix} = \boldsymbol{U}
\end{equation*}

其$\boldsymbol{LU}$分解存在且唯一.

\newpage
\subsection{列主元素三角分解法}
为了避免计算出的$u_{ii}$相对较小而放大舍入误差,在计算$u_{ii}$时先寻找绝对值最大的元作为主元再分解计算。

\begin{th4}
    对非奇异矩阵$\boldsymbol{A}$,存在排列阵$\boldsymbol{P}$,以及元素绝对值不大于1的单位下三角阵$\boldsymbol{L}$和上三角阵$\boldsymbol{U}$,使得:
    \begin{equation*}
        \boldsymbol{PA} = \boldsymbol{LU}
    \end{equation*}
\end{th4}

\subsection{平方根法}
\begin{th4}
    若$n$阶矩阵$\boldsymbol{A}$对称正定,则有唯一分解:
    \begin{equation}
        \boldsymbol{A} = \boldsymbol{LDL}^{T}
    \end{equation}
    其中,$\boldsymbol{L}$为单位下三角矩阵,$\boldsymbol{D}$为对角阵且对角元素全大于0.
\end{th4}

\begin{proof}
    由$\boldsymbol{A}$对称正定可知,$D_i$全大于0,对$\boldsymbol{U}$进行分解,有:
    \begin{equation*}
        \boldsymbol{A} = 
        \begin{bmatrix}
            1 & \ & \ & \ \\
            l_{21} & 1 & \ & \ \\
            \vdots & \vdots & \ddots & \ \\
            l_{n1} & l_{n2} & \cdots & 1 \\ 
        \end{bmatrix} \ 
        \begin{bmatrix}
            u_{11} & \      & \      & \      \\
            \      & u_{22} & \      & \      \\
            \      & \      & \ddots & \      \\
            \      & \      & \      & u_{nn} \\
        \end{bmatrix}
        \begin{bmatrix}
            1      & u_{12}/u_{11} & \cdots & u_{1n}/u_{11} \\
            \      & 1      & \cdots & u_{2n}/u_{22} \\
            \      & \      & \ddots & \vdots \\
            \      & \      & \      & 1      \\
        \end{bmatrix} \equiv \boldsymbol{LDU}_0
    \end{equation*}
    由$\boldsymbol{A}=\boldsymbol{A}^T$得:
    \begin{equation*}
        \boldsymbol{LDU}_0 = \boldsymbol{U}_0^T \boldsymbol{DL}^T
    \end{equation*}
    由于$\boldsymbol{LU}$分解具有唯一性,所以:
    \begin{equation*}
        \boldsymbol{U}_0^T = \boldsymbol{L},\quad \boldsymbol{L}^T = \boldsymbol{U}_0
    \end{equation*}
    故$\boldsymbol{A}$有唯一分解$\boldsymbol{A}=\boldsymbol{LDL}^T$.
\end{proof}

    若取$\overline{\boldsymbol{D}}=\mathrm{diag} (\sqrt{u_{11}},\sqrt{u_{22}},\dots,\sqrt{u_{nn}})$,则:
    \begin{equation*}
        \boldsymbol{A}=\boldsymbol{L}\overline{\boldsymbol{D}}^2 \boldsymbol{L}^T = (\boldsymbol{L\overline{D}})(\boldsymbol{L\overline{D}})^T \equiv \boldsymbol{GG}^T
    \end{equation*}
    
    其中$\boldsymbol{G}$为下三角矩阵,这种分解称为对称正定矩阵的\textbf{平方根分解(Cholesky分解)}

\begin{th4}
    若$n$阶矩阵$\boldsymbol{A}$对称正定,则有如下分解:
    \begin{equation}
        \boldsymbol{A} = \boldsymbol{GG}^{T}
    \end{equation}
    其中$\boldsymbol{G}$为下三角阵。若规定$\boldsymbol{G}$的对角元为正,则该分解唯一。
\end{th4}

记$\boldsymbol{G}=(g_{ij})_{n \times n}$,则平方根分解算法如下:
对$k=1,2,\dots,n$,有:
\begin{equation}
    \begin{cases}
        \ a_{kk} \leftarrow g_{kk} = \sqrt{a_{kk} - \sum\limits_{j=1}\limits^{k-1} g^{2}_{kj}} \\[5mm]
        \ a_{kj} \leftarrow g_{kj} = {\displaystyle \frac{a_{kj} - \sum\limits_{i=1}\limits^{j-1}g_{ki}g_{ji}}{g_{jj}}} \quad (k=j+1,j+2,\dots,n) 
    \end{cases}
\end{equation}

\begin{th4}
    若线性代数方程组$\boldsymbol{Ax} = \boldsymbol{b}$的系数矩阵$\boldsymbol{A}$对称正定,则用平方根法进行求解是稳定的。
\end{th4}


\subsection{三对角方程组的追赶法}
在建立三次样条插值函数、求微分方程数值解等问题中,常遇到三对角方程组$\boldsymbol{Ax} = \boldsymbol{g}$:
\begin{equation}
    \begin{bmatrix}
        b_1 & c_1       & \       & \      & \       \\
        a_2 & b_2       & c_2     & \      & \       \\
        \   & \ddots    & \ddots  & \ddots & \       \\
        \   & \         & a_{n-1} & b_{n-1}& c_{n-1} \\
        \   & \         & \       & a_{n}  & b_{n}   \\
    \end{bmatrix}\ 
    \begin{bmatrix}
        x_1 \\ x_2 \\ \vdots \\ x_{n-1} \\ x_n \\
    \end{bmatrix} = 
    \begin{bmatrix}
        g_1 \\ g_2 \\ \vdots \\ g_{n-1} \\ g_n \\
    \end{bmatrix}
\end{equation}

系数矩阵$\boldsymbol{A}$\textbf{严格对角占优},即满足每一行的主元的绝对值大于该行其他元素绝对值之和:
\begin{equation}
    \begin{cases}
        \ \left| b_1 \right| > \left| c_1 \right| \\
        \ \left| b_i \right| > \left| a_i \right| + \left| c_i \right| \\
        \ \left| b_n \right| > \left| a_n \right|  \\
    \end{cases}\quad (i=1,2,\dots,n-1)
\end{equation}
则$\boldsymbol{A}$非奇异,方程组$\boldsymbol{Ax}=\boldsymbol{g}$存在唯一解。

利用$\boldsymbol{LU}$分解:
\begin{equation*}
    \begin{bmatrix}
        b_1 & c_1       & \       & \      & \       \\
        a_2 & b_2       & c_2     & \      & \       \\
        \   & \ddots    & \ddots  & \ddots & \       \\
        \   & \         & a_{n-1} & b_{n-1}& c_{n-1} \\
        \   & \         & \       & a_{n}  & b_{n}   \\
    \end{bmatrix} = 
    \begin{bmatrix}
        1 & \ & \ & \ \\
        l_2 & 1 & \ & \ \\
        \ & \ddots & \ddots &\ \\
        \ & \ & l_n & 1 \\
    \end{bmatrix}\ 
    \begin{bmatrix}
        u_1 & d_1 & \   & \  &\  \\
        \   & u_2 & d_2 & \  &\  \\
        \   & \   & \ddots & \ddots & \ \\
        \   & \   & \   & \ddots & d_{n-1} \\
        \   & \   & \   & \  & u_n \\
    \end{bmatrix}
\end{equation*}

根据矩阵乘法记矩阵相等可得分解的计算公式:
\begin{equation}\label{thomas1}
    \begin{cases}
        \ d_i = c_i \quad (i=1,2,\dots,n-1) \\
        \ u_1 = b_1 \\
        \ l_i = a_i / u_{i-1},\ u_i = b_i - l_i d_{i-1} \quad (i=2,3,\dots,n)
    \end{cases}
\end{equation}

从而原将方程组$\boldsymbol{Ax}=\boldsymbol{g}$转化为求解方程组$\boldsymbol{Ly}=\boldsymbol{g}$和$\boldsymbol{Ux}=\boldsymbol{y}$,即:
\begin{align}
    & \begin{cases}
        \ y_1 = g_1 \\
        \ y_i = g_i - l_i y_{i-1} \quad (i=2,3,\dots,n)
    \end{cases} \label{thomas2}\\
    & \begin{cases}
        \ x_n = {\displaystyle \frac{y_n}{u_n}} \\[3mm]
        \ x_i = {\displaystyle \frac{y_i - c_i x_{i+1}}{u_i}} \quad (i=n-1,\dots,1)
    \end{cases} \label{thomas3}
\end{align}

求解式(\ref{thomas1})、式(\ref{thomas2})和式(\ref{thomas3})的方法称为\textbf{追赶法(Thomas方法)}。


\newpage
\section{向量范数、矩阵范数、条件数}
范数是为了度量向量和矩阵的尺度,例如在一维上用绝对值$\left|x-x_0\right|$的大小度量$x \rightarrow x_0$,是欧氏空间中向量长度的推广,为下一章研究$\boldsymbol{Ax}=\boldsymbol{b}$的解对$\boldsymbol{A}$和$\boldsymbol{b}$的敏感性以及迭代法求解的收敛性和迭代解的误差估计等问题。

\subsection{向量范数、矩阵范数、谱半径}
\begin{enumerate}
    \item \textbf{Def:}对任一向量$\boldsymbol{x}=(x_1,x_2,\dots,x_n)^T \in \mathbf{R}$,对应一个实值函数$N(\boldsymbol{x})=\Vert \boldsymbol{x} \Vert$,若满足下列性质:
    \begin{enumerate}
        \item $\Vert \boldsymbol{x} \Vert \geqslant 0,\ \Vert \boldsymbol{x} \Vert = 0 \Leftrightarrow \boldsymbol{x}=\boldsymbol{0} \in \mathbf{R}^n$(正定性)
        \item $\forall \alpha \in \mathbf{R},\ \Vert \alpha \boldsymbol{x} \Vert = \vert \alpha \vert \Vert \boldsymbol{x} \Vert$(齐次性)
        \item $\forall \boldsymbol{y} = (y_1,y_2,\dots,y_n)^T \in \mathbf{R}^n,\ \Vert \boldsymbol{x}+\boldsymbol{y} \Vert \leqslant \Vert \boldsymbol{x} \Vert + \Vert \boldsymbol{y} \Vert$(三角不等式)
    \end{enumerate}
    则称$N(\boldsymbol{x})=\Vert \boldsymbol{x} \Vert$为\textbf{向量$\boldsymbol{x}$的范数}.
    
    $\mathbf{R}^n$中常见的几种范数有:
    \begin{align}
        \Vert \boldsymbol{x} \Vert_1 \ &= \sum\limits_{i=1}\limits^{n} \vert x_i \vert \quad \text{\small (1-范数)}\\[3mm]
        \Vert \boldsymbol{x} \Vert_2 \ &= (\sum\limits_{i=1}\limits^{n} x_i^2)^{\frac{1}{2}} \quad \text{\small (2-范数)}\\[3mm]
        \Vert \boldsymbol{x} \Vert_\infty &= \max\limits_{1\leqslant i\leqslant n} \{\vert x_i \vert\} \quad \text{\small ($\infty$-范数)}
    \end{align}

    向量范数的等价性:设$\Vert \cdot \Vert_s,\ \Vert \cdot \Vert_t$为$\mathbf{R}^n$上向量$\boldsymbol{x}$的任意两种范数,则存在与$\boldsymbol{x}$无关的常数$m,M>0$使得:
    \begin{equation}
        m\Vert \boldsymbol{x}\Vert_s \leqslant \Vert \boldsymbol{x}\Vert_t \leqslant M\Vert \boldsymbol{x} \Vert_s, \quad \forall \boldsymbol{x} \in \mathbf{R}^n.
    \end{equation}

    \textbf{def:}设$\{ \boldsymbol{x}^{(k)} \}$为$\mathbf{R}^n$中一向量序列,$\boldsymbol{x}^{*} \in \mathbf{R}^n$,若:
    \begin{equation*}
        \lim_{k\rightarrow \infty} x_j^{(k)} = x_j^{*}\quad (j=1,2,\dots,n)
    \end{equation*}
    则称向量序列$\{ \boldsymbol{x}^{(k)} \}$收敛于$\boldsymbol{x}^*$,并有如下\textbf{th}成立:
    \begin{equation*}
        \lim_{k\rightarrow \infty} \boldsymbol{x}^{(k)} = \boldsymbol{x}^\ast \Leftrightarrow \lim_{k\rightarrow \infty} \Vert \boldsymbol{x}^{(k)} - \boldsymbol{x}^\ast \Vert = 0
    \end{equation*}


    \newpage
    \item \textbf{Def:}若矩阵$\boldsymbol{A} \in \mathbf{R}^{n\times n}$的非负实值函数$N(\boldsymbol{A}) = \Vert \boldsymbol{A} \Vert$满足以下条件:
    \begin{enumerate}
        \item $\Vert \boldsymbol{A} \Vert \geqslant 0,\ \Vert \boldsymbol{A} \Vert = 0 \Leftrightarrow \boldsymbol{A}=\boldsymbol{0} \in \mathbf{R}^{n\times n}$(正定性)
        \item $\forall \alpha \in \mathbf{R},\ \Vert \alpha \boldsymbol{A} \Vert = \vert \alpha \vert \Vert \boldsymbol{A} \Vert$(齐次性)
        \item $\forall\ \boldsymbol{A}, \boldsymbol{B} \in \mathbf{R}^{n \times n},\ \Vert \boldsymbol{A}+\boldsymbol{B} \Vert \leqslant \Vert \boldsymbol{A} \Vert + \Vert \boldsymbol{B} \Vert$(三角不等式)
        \item $\forall\ \boldsymbol{A}, \boldsymbol{B} \in \mathbf{R}^{n \times n},\ \Vert \boldsymbol{AB} \Vert \leqslant \Vert \boldsymbol{A} \Vert \ \Vert \boldsymbol{B} \Vert$
    \end{enumerate}
    则称$N(\boldsymbol{A}) = \Vert \boldsymbol{A} \Vert$为\textbf{矩阵$\boldsymbol{A}$的范数}.

    $\mathbf{R}^{n \times n}$中常见的几种范数有:
    \begin{align}
        \Vert \boldsymbol{A} \Vert_1 \ &= \max_{1 \leqslant j \leqslant n} \sum\limits_{i=1}\limits^{n} \vert a_{ij} \vert \quad \text{\small ($\boldsymbol{A}$的列范数)}\\[3mm]
        \Vert \boldsymbol{A} \Vert_\infty &= \max_{1 \leqslant i \leqslant n} \sum\limits_{j=1}\limits^{n} \vert a_{ij} \vert \quad \text{\small ($\boldsymbol{A}$的行范数)}\\[3mm]
        \Vert \boldsymbol{A} \Vert_2 \ &= \sqrt{\lambda_{\max}(\boldsymbol{A}^T\boldsymbol{A})} \quad \text{\small ($\boldsymbol{A}$的谱范数)}\\[4mm]
        \Vert \boldsymbol{A} \Vert_\mathrm{F} \ &= (\sum\limits_{i,j=1}\limits^{n}a_{ij}^{2})^{\frac{1}{2}} \quad \text{\small ($\boldsymbol{A}$的F范数)}
    \end{align}
    其中$\boldsymbol{A}=(a_{ij})_{n \times n},\ \lambda_{\max}(\boldsymbol{A}^T\boldsymbol{A})$为$\boldsymbol{A}^T\boldsymbol{A}$的最大特征值。

    向量和矩阵范数的相容性:
    \begin{equation*}
        \Vert \boldsymbol{Ax} \Vert_s \leqslant \Vert \boldsymbol{A} \Vert \ \Vert \boldsymbol{x} \Vert_s \quad \forall \boldsymbol{A} \in \mathbf{R}^{n \times n}, x \in \mathbf{R}^n
    \end{equation*}

    \textbf{def:}设$\{ \boldsymbol{A}^{(k)} \}$为$\mathbf{R}^{n\times n}$中的矩阵序列,$\boldsymbol{A} \in \mathbf{R}^{n\times n}$,若:
    \begin{equation*}
        \lim_{k\rightarrow \infty} a_{ij}^{(k)} = a_{ij} \quad (i,j=1,2,\dots,n)
    \end{equation*}
    则称向量序列$\{ \boldsymbol{A}^{(k)} \}$收敛于$\boldsymbol{A}$,并有如下\textbf{th}成立:
    \begin{equation*}
        \lim_{k\rightarrow \infty} \boldsymbol{A}^{(k)} = \boldsymbol{A} \Leftrightarrow \lim_{k\rightarrow \infty} \Vert \boldsymbol{A}^{(k)} - \boldsymbol{A} \Vert = 0
    \end{equation*}


    \newpage
    \item \textbf{Def:}设$\boldsymbol{A} \in \mathbf{R}^{n\times n}$的特征值为$\lambda_i\ (i=1,2,\dots,n)$,定义矩阵$\boldsymbol{A}$的\textbf{谱半径}为:
    \begin{equation*}
        S(\boldsymbol{A}) = \max_{1\leqslant i \leqslant n}\{\vert\lambda_i\vert\}
    \end{equation*}
    根据特征值$\boldsymbol{Ax}=\lambda\boldsymbol{x}$知$\vert \lambda \vert \ \Vert \boldsymbol{x}\Vert = \Vert\boldsymbol{Ax}\Vert \leqslant \Vert\boldsymbol{A}\Vert \ \Vert\boldsymbol{x}\Vert$,且$\boldsymbol{x} \neq 0$,所以:
    \begin{equation*}
        S(\boldsymbol{A}) \leqslant \Vert\boldsymbol{A}\Vert
    \end{equation*}

    \textbf{Th:}设任意$n$阶矩阵$\boldsymbol{F}$满足$\Vert\boldsymbol{F}\Vert < 1$,则$\boldsymbol{I}\pm\boldsymbol{F}$非奇异($\boldsymbol{I}$为单位矩阵),且:
    \begin{equation*}
        \Vert(\boldsymbol{I}\pm\boldsymbol{F})\Vert \leqslant \frac{1}{1-\Vert\boldsymbol{F}\Vert}
    \end{equation*}

    \textbf{Th:}设$\boldsymbol{A}\in\mathbf{R}^{n\times n}$,由$\Vert\boldsymbol{A}\Vert$的各次幂所组成的矩阵序列:
    \begin{equation*}
         \boldsymbol{A}^0,\boldsymbol{A},\boldsymbol{A}^2,\cdots,\boldsymbol{A}^k,\cdots
    \end{equation*}
    其中$\boldsymbol{A}^0 = \boldsymbol{I}$,该序列收敛于零矩阵(即$\lim\limits_{k\rightarrow\infty}\boldsymbol{A}^k = 0$)的充要条件为:
    \begin{equation*}
        S(\boldsymbol{A}) < 1
    \end{equation*}
\end{enumerate}

\subsection{矩阵条件数及方程组性态*}
略

% \end{document}