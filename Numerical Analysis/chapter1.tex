% \documentclass[a4paper, 11pt, oneside, onecolumn, titlepage]{ctexrep}

% \usepackage{makeidx,amsmath}
% \usepackage{unicode-math}
% \usepackage{setspace}
% \usepackage{amsthm}
% \usepackage{geometry}
% \usepackage{fancyhdr}
% \usepackage{tikz}

% \geometry{left=25mm,right=25mm,top=30mm,bottom=30mm}
% \setmathfont{XITS Math}

% \linespread{1.5}
% \title{Numerical Integration \& Numerical Differentiation}
% \author{Brassicaaa}
% \date{\today}

% \setlength{\headheight}{27pt}

% \begin{document}
% \pagestyle{fancy}
% \lhead{Notes for Numerical Analysis}
% \setcounter{page}{1}
% \pagenumbering{arabic}
\chapter{绪论}

	\textbf{现代研究方法包括:理论、实践、科学计算}

	函数 $f(x)$ 在 $x_0$ 处的 $n$ 次 Taylor 多项式展开:
	\begin{align}
	P_{n}(x) &= \sum_{k=0}^n \frac{f^{(k)}(x_0)}{k!} (x-x_0)^k \notag \\
	&= f(x_{0}) + \frac{f^{\prime}(x_0)}{1!}(x-x_0)+\dots + \frac{f^{(n)}(x_0)}{n!}(x-x_0)^n
	\end{align}

	误差:
	\begin{align}
	R_n(x) &= f(x) - P_n(x) \notag \\
	&= \frac{f^{(n+1)}(\xi)}{(n+1)!} (x-x_0)^{n+1}
	\end{align}


\section{误差}
	\textbf{误差的来源:模型误差、观测误差、截断误差、舍入误差}
	
	真实值:$x$ 
	
	测量值:$x^*$
	
	误差:$e^* = x^* - x$,$e^* > 0$ 近似值偏大,称为{\em 强近似值},$e^* < 0$ 近似值偏小,称为{\em 弱近似值}
	
	误差限:$\varepsilon^*$,$\left| e^*\right| = \left| x^* - x\right| \le \varepsilon^*$
	
	\begin{spacing}{2.0}
	相对误差:$e^*_r = \displaystyle{\frac{e^*}{x}\approx \frac{e^*}{x^*}}$ ($e_r^*$较小)
	
	相对误差限:$\varepsilon^*_r = \displaystyle{\frac{\varepsilon^*}{\left| x^* \right|}}$
	\end{spacing}\newpage
	
	$n$位有效数字的标准写法:
	\begin{equation}
	x^* = \pm 10^m \times (a_1 +a_2\times 10^{-1}+\dots + a_n\times 10^{-(n-1)})
	\end{equation}
	
	其误差限:
	\begin{equation}
	\varepsilon^* = \left| x^* - x\right| \le \frac{1}{2} \times 10^{m-n+1}
	\end{equation}
	
	\newtheorem{mythm}{Theorem}[chapter]
	\begin{mythm}
	对具有 $n$ 位有效数字的 $x^*$其相对误差限为:
	\begin{equation}
	\varepsilon^*_r \le \frac{1}{2a_1} \times 10^{-(n-1)}
	\end{equation}
	\end{mythm}

	定理1.1表明,有效数字位数越多,相对误差限越小。
	
	\textbf{和的误差是误差之和,差的误差是误差之差}
	
	\textbf{乘积的相对误差是各乘数的相对误差之和,商的相对误差是被除数与除数的相对误差之差}

	\begin{proof}{\em
	将 $x^*$ 的误差 $e^* = x^* - x$ 看作是对 $x$ 的微分:
	\begin{equation}
	\mathrm{d} x = x^* - x \notag
	\end{equation}
	
	则 $x^*$ 的相对误差就为对数函数的微分:
	\begin{equation}
	e^*_r = \frac{x^* - x}{x} = \frac{\mathrm{d}x}{x}=\mathrm{dln}x \notag
	\end{equation}
	
	设 $u=xy$,则 $\mathrm{ln}u = \mathrm{ln}x + \mathrm{ln}y$,故:
	$$
	\mathrm{dln}u = \mathrm{dln}x + \mathrm{dln}y \notag 
	$$}
	\end{proof}
	\newpage
	
\section{注意事项}
	\begin{enumerate}
		\item \textbf{避免两个相近的数相减}
		
			设 $u=x-y$,则相对误差:
			$$
			\mathrm{dln}u = \frac{\mathrm{d}x -\mathrm{d}y}{x-y} \notag
			$$
			如果下 $x$ 和 $y$ 很接近,就会导致 $u$ 的相对误差很大。
		
		\begin{enumerate}
		\begin{spacing}{2.2}
			\item 当 $x \to 0$,$\displaystyle{\frac{1-\mathrm{cos}x}{\mathrm{sin}x} \rightleftharpoons \frac{\mathrm{sin}x}{1+\mathrm{cos}x}}$ 
			
			\item 当 $(x_1 - x_2) \to 0$,$\displaystyle{(\mathrm{lg}x_1 - \mathrm{lg}x_2) \rightleftharpoons \mathrm{lg}\frac{x_1}{x_2}}$
			
			\item 当 $x \to \infty$,$\displaystyle{\sqrt{x+1}-\sqrt{x}\rightleftharpoons \frac{1}{\sqrt{x+1}+\sqrt{x}}}$
			
			\item 当 $f(x) \approx f(x^*)$时,用 Taylor 级数展开:
			{\setlength\abovedisplayskip{0cm}
			\setlength\belowdisplayskip{0cm}
			$$
			f(x)-f(x^*) = f^{\prime}(x^*)(x-x^*) + \frac{f^{\prime \prime}(x^*)}{2!}(x-x^*)^2 + \dots \notag
			$$}
		\end{spacing}
		\end{enumerate}
		\item \textbf{防止大数吃掉小数}
		\item \textbf{避免除数绝对值远远小于被除数绝对值的除法}
		\item \textbf{简化运算步骤,较少运算次数}
	\end{enumerate}

% \end{document}